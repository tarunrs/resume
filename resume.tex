%% start of file `template.tex'.
%% Copyright 2006-2012 Xavier Danaux (xdanaux@gmail.com).
%
% This work may be distributed and/or modified under the
% conditions of the LaTeX Project Public License version 1.3c,
% available at http://www.latex-project.org/lppl/.


\documentclass[11pt,a4paper,sans]{moderncv}   % possible options include font size ('10pt', '11pt' and '12pt'), paper size ('a4paper', 'letterpaper', 'a5paper', 'legalpaper', 'executivepaper' and 'landscape') and font family ('sans' and 'roman')

% moderncv themes
\moderncvstyle{classic}                        % style options are 'casual' (default), 'classic', 'oldstyle' and 'banking'
\moderncvcolor{blue}                          % color options 'blue' (default), 'orange', 'green', 'red', 'purple', 'grey' and 'black'
%\renewcommand{\familydefault}{\sfdefault}    % to set the default font; use '\sfdefault' for the default sans serif font, '\rmdefault' for the default roman one, or any tex font name
%\nopagenumbers{}                             % uncomment to suppress automatic page numbering for CVs longer than one page

% character encoding
%\usepackage[utf8]{inputenc}                  % if you are not using xelatex ou lualatex, replace by the encoding you are using
%\usepackage{CJKutf8}                         % if you need to use CJK to typeset your resume in Chinese, Japanese or Korean

% adjust the page margins
\usepackage[scale=0.75]{geometry}
%\setlength{\hintscolumnwidth}{3cm}           % if you want to change the width of the column with the dates
%\setlength{\maketitlenamewidth}{10cm}        % for the 'classic' style, if you want to force the width allocated to your name and avoid line breaks. be careful though, the length is normally calculated to avoid any overlap with your personal info; use this at your own typographical risks...

% personal data
\firstname{Tarun Ronur}
\familyname{Sasikumar}
\address{Total experience: 8 years}{Bangalore, India}    % optional, remove the line if not wanted
\mobile{+91~(819)~741~1464}                     % optional, remove the line if not wanted
\email{tarunrs@gmail.com}                        % optional, remove the line if not wanted
\homepage{https://github.com/tarunrs}
% to show numerical labels in the bibliography (default is to show no labels); only useful if you make citations in your resume
%\makeatletter
%\renewcommand*{\bibliographyitemlabel}{\@biblabel{\arabic{enumiv}}}
%\makeatother

% bibliography with mutiple entries
%\usepackage{multibib}
%\newcites{book,misc}{{Books},{Others}}
%----------------------------------------------------------------------------------
%            content
%----------------------------------------------------------------------------------
\begin{document}
%\begin{CJK*}{UTF8}{gbsn}                     % to typeset your resume in Chinese using CJK
\maketitle
\section{Objective}
\cvitem{}{To obtain a Full-time position in data science and software engineering}
\section{Education}
\cventry{2011 -- 2013}{Ohio State University}{Columbus, OH}{}{}{MS in Computer Science and Engineering. GPA: 3.7/4.0}
\cventry{2004 -- 2008}{Vishweswaraiah Technological University}{Belgaum, India}{}{}{BE in Computer Science and Engineering.}
\section{Professional Work Experience}
% \subsection{Vocational}
\cventry{2019 -- Prsent}{Tech lead - Data Science}{Zoomtail}{Bangalore, India}{}{ Leading the entire Data Science effort.\newline
\emph{Highlights:}
\begin{itemize}
\item \textbf{} Implemented a visual search system.
\item \textbf{} Working on building dashboards for different stakeholders.
\end{itemize}}
\cventry{2016 -- 2019}{Data Scientist}{Crediwatch}{Bangalore, India}{}{ Led the Data Science effort. Responsible for the entire data science pipeline, including system architecture and mentoring of team members.\newline
\emph{Highlights:}
\begin{itemize}
\item \textbf{} Implemented a distributed crawling infrastructure capable of crawling and indexing 100+ pages/minute per bot (using scrapy, frontera, Elasticsearch, Kafka and HBase). 
\item \textbf{} Designed and developed a real-time news search system with sentiment analysis, named entity extraction and clustering of news stories. 
\item \textbf{} Built a system to detect and generate Breaking News.
\item \textbf{} Improved the precision of a date extractor library by 20\%. 
\item \textbf{} Built multiple classification models to tag news articles with metadata. 
\item \textbf{} Invented a new pre-clustering algorithm based on Canopy clustering algorithm that also optimizes on the density of the clusters. 
\item \textbf{} Built a Lead Generation system using machine learning(kNN) and graph mining(Pagerank) techniques with a potential to improve conversion rates by 2-7x
\item \textbf{} Built an interactive visualization framework for text such as Adverse News articles similar to quid.com(sigma.js, Elasticsearch)
\item \textbf{} Bootstrapped a dashboard framework for analytics team (crontab / python / elasticsearch / HTML / Altair/ Vega)
\end{itemize}}
\cventry{2013 -- 2015}{Software Engineer}{Epic Systems Corp.}{Verona, Wisconsin}{}{Responsible for design and development of frontend and backend modules used to document Oncology related information in the EMR. Instrumental in adding support to interface with third-party Radiation Oncology systems. }
\cventry{2008 -- 2011}{Software Engineer}{Atlantis Computing}{Bangalore, India}{}{Responsible for design, development and maintenance of user profile and application virtualization components in the Atlantis ILIO product stack. Built expertise in the core filesystem level and driver development of both Windows and Linux platforms.}
\section{Research Projects}
\cventry{}{\small tCTR - Network attention predictor for tweets}{}{Prof. Srinivasan Parthasarathy}{}{Studied the effect of language syntax and sentiments in sparse data like tweets to predict network attention. Showed that using these lower dimensional language features contributed to network attention more than word based features or novelty of the subject. }
\cventry{}{\small Realtime Analysis of Tweets}{}{Prof. Arnab Nandi}{}{Implemented a system for real-time analysis of sentiments and surfacing of contextual tweets related to political events like the Presidential debates. The supervised model achieves a precision of 85\% with a response time of less than 200ms on the 1\% Twitter firehose. }
\cventry{}{\small Link Prediction - KDD Cup 2012}{}{ Prof. Srinivasan Parthasarathy}{}{Part of a team of 3 that took part in the KDD Cup link prediction challenge in social networks. Wrote multiple scripts to extract semantic and network features from the social graph which helped achieve a weighted average precision of 81\%.}
\cventry{}{\small pSnipSuggest}{}{Prof. Arnab Nandi}{}{Implemented a system that provides on-the-go, context-aware assistance in the SQL composition process, based on SnipSuggest by  Khoussainova et al. Increased the average precision by 21\% and the response time by 30\% compared to the original implementation. Written in Python}
\section{Open Source and Personal Projects}
\cventry{}{\small jamMm.in}{}{}{}{\small Co-founded jamMm.in, an online music collaboration application. Served as the User Experience developer. Wrote the framework for mixing multiple tracks as well as the library to generate waveform images for the tracks. Written in Python and Ruby.}
\cventry{}{\small Movie Showtimes}{}{}{}{\small Developed an Android application for Movie showtime/theatre details based on the user's location. Downloaded over 36,000 times in the Android Market. Written in Java }
\hfill\scriptsize{These projects are listed with source code at: \href{https://github.com/tarunrs}{https://github.com/tarunrs}}
\normalsize
\section{Technical Skills}
\cvitem{Languages}{\small Experience: Python, C\newline Intermediate: HTML/CSS, Javascript, Java, C++\newline Familiar with distributed/parallel processing systems like Spark, MapReduce/Hadoop ecosystem}
\cvitem{Datastores}{\small MongoDb, HBase, Elasticsearch, MySQL, Postgres }
\cvitem{Message Queues}{\small RabbitMQ, Kafka, Redis}
\cvitem{Data Science}{\small scikit-learn, numpy, nltk, spacy, gensim, pandas, CoreNLP, Keras, fast.ai, altair}
\cvitem{Tools}{\small Git, SVN, Balsamiq, Weka.\newline }
{ } 
\end{document}
